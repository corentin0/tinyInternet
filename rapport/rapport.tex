\documentclass[a4paper,12pt]{report} % Format du papier, type de document
 
\usepackage[utf8]{inputenc} % Permet de tapper les accents tels quels
\usepackage[T1]{fontenc} % Permet l'utilisation d'accents
\usepackage[french]{babel} % Dit que le document est en français
\usepackage{amsfonts} % Différents packages... Lire la doc....
\usepackage{amsmath}
\usepackage{listings}
\usepackage{color}
\usepackage{graphicx} %affichage d'image
\usepackage{moreverb}
\usepackage[colorlinks=true,linkcolor=black]{hyperref} %lien hypertexte
\usepackage{fullpage}
\usepackage{fancyhdr}
\pagestyle{fancy}
\renewcommand{\thesection}{\arabic{section}}
\setcounter{tocdepth}{5}
\setcounter{secnumdepth}{5}

\renewcommand{\headrulewidth}{0pt} 
\fancyhead{} %retire en tete

\renewcommand{\footrulewidth}{1pt} %bas de page
\fancyfoot[C]{\textbf{\thepage}} 
\fancyfoot[L]{Rapport Réseau}
\fancyfoot[R]{DUCRUET C. - GHEYSEN J.}

\title{Réseau II \\ Projet Tiny Internet\\ Rapport}
\author{UMONS\\Faculté des Sciences\\MA 1 Sciences - Informatique\\DUCRUET Corentin \\GHEYSEN Jérémy}
\date{Année académique\\2016 - 2017}





%\begin{figure}[!h] %on ouvre l'environnement figure
%		\centering
%		\includegraphics[width=108mm,height=65mm]{impulsion.png}
%	\end{figure} 

%\begin{figure}%[H] si on veut que l image soit a cet endroit
%	\centering
%	\includegraphics[width=108mm,height=65mm]{./scr/logo}
%	\caption{Icône}
%	\label{fig:Icône}
%\end{figure}

\begin{document} 
\maketitle
\newpage 
\tableofcontents 
\newpage
\raggedright

\section{Introduction}
Le but de ce projet était de modéliser une interconnexion de réseaux en utilisant C-BGP. Il nous était demandé de réaliser une configuration complète de plusieurs routeurs appartenant à divers domaines.

\section{Choix de design}
La topologie que nous devions implémenter était la suivante:
\begin{figure}[!h] %on ouvre l'environnement figure
		\centering
		\includegraphics[width=120mm,height=80mm]{topologie}
		\label{topo}
		\caption{Topology}
	\end{figure} 

Dans cette topologie, \textit{iCompany, Spring, BigCarrier} sont des réseaux commerciaux,  \textit{Abilene, GEANT, BELNET, CERN} sont des réseaux de non profits, \textit{UCLA, UCL, Ulg} sont des réseaux universitaires.\\
Les réseaux \textit{iCompany, CERN} et les réseaux universitaires ne fournissent pas de services de transit pour les autres réseaux. A l'opposé, on retrouve les réseaux \textit{BigCarrier} et \textit{Spring} dont le but est de faire le transit de paquets. \textit{Abilene, GEANT, BELNET} sont quant à eux des réseaux de transit spéciaux financés par le gouvernement et qui ne fournissent de la connectivité qu'aux réseaux  de non profits.\\
Nous avons définit le poids IGP de chaque lien grâce au temps de transmission de ce lien. Chaque domaine sera identifié grâce à son numéro d'AS.\\

Toute les routes inter-domaines ont été ajoutées statiquement. Le routage intra-domaine est réalisé grâce à IGP et iBGP.\\

En ce qui concerne les filtres, de manière générale, nous avons opté pour la solution suivante:\\
soit 3 AS : AS1, AS2, AS3 (voir figure \ref{as}). AS1 est fournisseur de AS2 et AS3 et il y a un lien de peering entre AS2 et AS3. Plaçons nous dans l'AS2, s'il reçoit des routes venant de AS3 il ne vas pas les annoncer à AS1 car il pourrait les utiliser ce qui est contraire au politique de routage. De même, s'il reçoit des routes de AS1, il ne va pas les annoncer à l'AS3. Nous avons comme cas particulier que Abilene et GEANT ne transmettent pas les routes de Spring et BigCarrier car ce sont des réseaux de non-profit. Il y a une exception pour GEANT qui transmet les routes au CERN sinon il n'a pas de connectivité aux AS Spring, BigCarrier et iCompany.
Pour respecter les liens customer-provider et peering, nous avons utilisé les communautés (1 pour customer-provider et 2 pour peering). Il est donc plus facile de choisir quelles routes transmettre en fonction de la relation avec le voisin.
\begin{figure}[!h] %on ouvre l'environnement figure
		\centering
		\includegraphics[width=120mm,height=80mm]{as}
		\label{as}
		\caption{Exemple}
	\end{figure} 
	
\section{Problèmes}
Nous avons rencontré un problème que nous n'avons su résoudre. Celui-ci consiste en l'injoignabilité des préfixes 2.2.1.0/24 et 1.1.0.0/24 depuis n'importe quel autre routeur d'autres sous-réseaux. Les différentes tentatives pour joindre ces deux préfixes aboutissent toute à un bouclage infini lors du calcul de la route. 

\section{Conclusion}
Ce projet nous a permis de mieux comprendre certains principes de BGP ainsi que nous apprendre quelques base dans la modélisation d'une interconnexion de réseaux.



\end{document}